\documentclass[11pt,oneside]{book}

% Shared style for SaulPOS manuals.
\usepackage{fontspec}
\usepackage{polyglossia}
\usepackage[a4paper,margin=2.5cm,headheight=15pt]{geometry}
\usepackage{microtype}
\usepackage{graphicx}
\usepackage{xcolor}
\usepackage{booktabs}
\usepackage{longtable}
\usepackage{tabularx}
\usepackage{array}
\usepackage{enumitem}
\usepackage{hyperref}
\usepackage{fancyhdr}
\usepackage{titlesec}
\usepackage{tcolorbox}
\tcbuselibrary{breakable}
\usepackage{caption}
\usepackage{etoolbox}
\usepackage{float}
\usepackage{multirow}
\usepackage{makecell}
\usepackage{lastpage}

\setmainfont{TeX Gyre Pagella}
\setsansfont{TeX Gyre Heros}
\setmonofont{Latin Modern Mono}

\definecolor{SaulBlue}{HTML}{0B3A53}
\definecolor{SaulTeal}{HTML}{0F766E}
\definecolor{SaulNote}{HTML}{1D4ED8}
\definecolor{SaulWarn}{HTML}{B45309}
\definecolor{SaulTip}{HTML}{166534}

\hypersetup{
  colorlinks=true,
  linkcolor=SaulBlue,
  urlcolor=SaulTeal,
  citecolor=SaulBlue,
  pdfauthor={Saul Technologies}
}

\setlist[itemize]{leftmargin=1.2cm,itemsep=0.3em,topsep=0.4em}
\setlist[enumerate]{leftmargin=1.2cm,itemsep=0.3em,topsep=0.4em}

\captionsetup{font=small,labelfont=bf}

\pagestyle{fancy}
\fancyhf{}
\fancyhead[L]{\nouppercase{\leftmark}}
\fancyhead[R]{\thepage}
\renewcommand{\headrulewidth}{0.4pt}

\titleformat{\chapter}[display]
  {\normalfont\bfseries\Huge\color{SaulBlue}}
  {\filleft\Large\chaptername\ \thechapter}
  {1ex}
  {\filright}
  [\vspace{1ex}\titlerule]

\setlength{\parskip}{0.5em}
\setlength{\parindent}{0pt}

\newcolumntype{Y}{>{\raggedright\arraybackslash}X}

\setdefaultlanguage{spanish}
\setotherlanguage{english}
% Spanish labels and key terms.
\newcommand{\SaulPOS}{\textsc{SaulPOS}}
\newcommand{\CompanyName}{Saul Technologies}
\newcommand{\ManualTitleText}{Manual de Usuario SaulPOS}
\newcommand{\ManualSubtitleText}{Guía operativa para ventas, inventario y administración}
\newcommand{\TipWord}{Consejo}
\newcommand{\WarningWord}{Advertencia}
\newcommand{\NoteWord}{Nota}
\newcommand{\RulesWord}{Regla}
\newcommand{\DetailWord}{Detalle}
\newcommand{\ExampleWord}{Ejemplo}
\newcommand{\ResultWord}{Resultado}
\newcommand{\GlossaryTermHeader}{Término}
\newcommand{\GlossaryDefinitionHeader}{Definición}

\newcommand{\TermProduct}{producto}
\newcommand{\TermService}{servicio}
\newcommand{\TermCategory}{categoría}
\newcommand{\TermDiscount}{descuento}
\newcommand{\TermTax}{impuesto}
\newcommand{\TermInventory}{inventario}
\newcommand{\TermReturn}{devolución}
\newcommand{\TermExchange}{cambio}
\newcommand{\TermCashier}{cajero}
\newcommand{\TermSupervisor}{supervisor}
\newcommand{\TermAdmin}{administrador}

% Shared macros for both manuals.
\newcommand{\SaulLogo}[1][0.28\textwidth]{%
  \IfFileExists{../images/logo/saul-technologies-logo.png}{%
    \includegraphics[width=#1]{../images/logo/saul-technologies-logo.png}%
  }{%
    \fbox{\begin{minipage}[c][3.2cm][c]{0.60\textwidth}
      \centering
      Saul Technologies\\
      Logo file missing:\\
      \texttt{\detokenize{../images/logo/saul-technologies-logo.png}}
    \end{minipage}}%
  }%
}

\newcommand{\SaulCoverPage}[4]{%
  \begin{titlepage}
    \thispagestyle{empty}
    \begin{center}
      \vspace*{1.7cm}
      \SaulLogo[0.30\textwidth]\\[1.3cm]
      {\Huge\bfseries #1\par}
      \vspace{0.4cm}
      {\Large #2\par}
      \vfill
      {\large \CompanyName\par}
      \vspace{0.2cm}
      {\large #3\par}
      {\large #4\par}
    \end{center}
  \end{titlepage}
}

\newtcolorbox{tipbox}{
  breakable,
  colback=SaulTip!8!white,
  colframe=SaulTip!65!black,
  title=\TipWord,
  fonttitle=\bfseries
}

\newtcolorbox{warningbox}{
  breakable,
  colback=SaulWarn!8!white,
  colframe=SaulWarn!70!black,
  title=\WarningWord,
  fonttitle=\bfseries
}

\newtcolorbox{notebox}{
  breakable,
  colback=SaulNote!6!white,
  colframe=SaulNote!70!black,
  title=\NoteWord,
  fonttitle=\bfseries
}

\newcommand{\SaulFigure}[3]{%
  \begin{figure}[H]
    \centering
    \IfFileExists{#1}{%
      \includegraphics[width=0.92\linewidth]{#1}%
    }{%
      \fbox{\begin{minipage}[c][0.30\textheight][c]{0.88\linewidth}
        \centering
        \textbf{Screenshot Placeholder}\\[0.6em]
        \texttt{\detokenize{#1}}
      \end{minipage}}%
    }
    \caption{#2}
    \label{#3}
    \vspace{0.4em}

    {\footnotesize\texttt{[FIGURE: \detokenize{#1} - "#2"]}}
  \end{figure}
}

\newenvironment{SaulGlossary}{%
  \begin{longtable}{p{0.30\textwidth}p{0.64\textwidth}}
  \toprule
  \textbf{\GlossaryTermHeader} & \textbf{\GlossaryDefinitionHeader}\\
  \midrule
  \endhead
}{%
  \bottomrule
  \end{longtable}
}

\newcommand{\GlossaryEntry}[2]{\textbf{#1} & #2\\\addlinespace}


\newcommand{\ManualVersion}{1.0.0}
\newcommand{\SoftwareVersion}{[Completar: versión de SaulPOS]}
\newcommand{\ManualDate}{\today}
\newcommand{\ShotPath}[1]{../images/placeholders/es/#1}
\newcommand{\ShotRef}[1]{\texttt{\detokenize{es/#1}}}

\begin{document}

\frontmatter
\SaulCoverPage{\ManualTitleText}{\ManualSubtitleText}{Versión del manual: \ManualVersion}{Fecha: \ManualDate}

\thispagestyle{empty}
\vspace*{3cm}
{\Large \textbf{Derechos de autor}}\\[0.8em]
\SaulPOS{} y su documentación son propiedad de \CompanyName.\\
Todos los derechos reservados. Ninguna parte de este documento puede reproducirse sin autorización escrita, excepto para uso interno de capacitación en el negocio licenciatario.\\[1em]
Primera edición interna.
\clearpage

\thispagestyle{empty}
{\Large \textbf{Página de versión}}\\[1.0em]
\begin{tabularx}{\textwidth}{>{\bfseries}p{0.36\textwidth}Y}
Versión del manual & \ManualVersion \\
Versión del software & \SoftwareVersion \\
Fecha de publicación & \ManualDate \\
Idioma & Español \\
Audiencia & Cajeros, supervisores, administradores y dueños de negocio \\
\end{tabularx}
\clearpage

\tableofcontents
\listoffigures
\clearpage

\mainmatter

\chapter{Introducción}\label{ch:intro}
\section{¿Qué es \SaulPOS?}
\SaulPOS{} es un sistema de punto de venta diseñado para registrar ventas, controlar inventario, gestionar clientes y producir reportes operativos diarios. Su enfoque es práctico: reducir errores en caja y dar trazabilidad a cada movimiento.

\section{¿Para quién es este manual?}
Este libro está orientado a:
\begin{itemize}
  \item Personas cajeras con conocimientos básicos de computadora.
  \item Supervisores que validan descuentos, devoluciones y cierres.
  \item Administradores responsables de catálogos, impuestos y usuarios.
  \item Dueños de negocio que revisan indicadores y reglas de operación.
\end{itemize}

\section{Beneficios clave}
\begin{itemize}
  \item Flujo de venta repetible y auditable.
  \item Reglas claras para descuentos, impuestos y devoluciones.
  \item Reducción de pérdidas por diferencias de inventario.
  \item Información consolidada para toma de decisiones.
\end{itemize}

\section{Visión general del sistema}
\SaulFigure{\ShotPath{introduccion_panel_principal_01.png}}{Pantalla: Panel principal del sistema}{fig:intro-panel}

\begin{notebox}
Aunque las pantallas reales pueden variar por versión o configuración, los principios operativos de este manual se mantienen.
\end{notebox}

\chapter{Conceptos Fundamentales}\label{ch:conceptos}
\section{Productos y servicios}
Un \TermProduct{} es un artículo físico que afecta inventario. Un \TermService{} no descuenta stock, pero sí impacta ventas e impuestos si está gravado.

\section{Categorías y subcategorías}
Las categorías agrupan productos para facilitar búsqueda, promociones y reportes. Ejemplo: \emph{Bebidas} $\rightarrow$ \emph{Jugos}.

\section{Clientes}
El registro de clientes permite historial de compra, notas y campañas comerciales. Si el módulo de crédito está habilitado por el administrador, también permite cuentas por cobrar.

\section{Precio, costo y margen}
\begin{longtable}{p{0.28\textwidth}p{0.66\textwidth}}
\toprule
\textbf{\RulesWord} & \textbf{\DetailWord}\\
\midrule
\endhead
Precio de venta & Monto cobrado al cliente antes o después de impuestos, según configuración.\\
Costo & Valor interno usado para utilidad y valoración de inventario.\\
Margen bruto & Relación entre precio y costo. No reemplaza la utilidad neta.\\
\bottomrule
\end{longtable}

\section{Descuentos}
\subsection{Porcentaje vs monto fijo}
\begin{longtable}{p{0.30\textwidth}p{0.30\textwidth}p{0.34\textwidth}}
\toprule
\textbf{Tipo} & \textbf{\ExampleWord} & \textbf{\ResultWord}\\
\midrule
\endhead
Porcentaje & 10\% sobre \$50.00 & Descuento \$5.00, nuevo subtotal \$45.00.\\
Monto fijo & \$5.00 sobre \$50.00 & Nuevo subtotal \$45.00.\\
\bottomrule
\end{longtable}

\subsection{Por artículo vs por ticket}
\begin{itemize}
  \item \textbf{Por artículo}: aplica solo a la línea seleccionada.
  \item \textbf{Por ticket}: aplica al total de la venta.
\end{itemize}

\subsection{Reglas de acumulación y redondeo}
\begin{warningbox}
Regla recomendada por defecto: no acumular dos descuentos manuales sobre la misma línea. Si su operación requiere acumulación, valide el impacto con su supervisor.
\end{warningbox}

\begin{tipbox}
Defina una única política de redondeo (por línea o al total) y úsela siempre. Cambiarla a mitad de mes complica conciliaciones.
\end{tipbox}

\SaulFigure{\ShotPath{conceptos_descuento_item_01.png}}{Pantalla: Aplicación de descuento por artículo}{fig:conceptos-descuento}

\section{Impuestos}
\begin{itemize}
  \item \textbf{Incluido}: el precio visible ya contiene impuesto.
  \item \textbf{Excluido}: el impuesto se agrega al final.
  \item \textbf{Exenciones}: disponibles si están habilitadas por el administrador y justificadas por normativa local.
\end{itemize}

\section{Inventario}
Conceptos base:
\begin{itemize}
  \item \textbf{Stock disponible}: unidades vendibles.
  \item \textbf{Ajuste}: corrección por merma, conteo o error.
  \item \textbf{Recepción}: ingreso desde compra a proveedor.
  \item \textbf{Transferencia}: movimiento entre sucursales, si está habilitado.
\end{itemize}

\section{Métodos de pago}
Caja, tarjeta y transferencia son métodos comunes. Si la configuración lo permite, puede dividir el cobro entre varios métodos.

\section{Devoluciones y reembolsos}
El plazo de devolución es configurable. Siempre documente motivo y referencia del comprobante.

\section{Roles y permisos}
\begin{itemize}
  \item \TermCashier: vende y cobra.
  \item \TermSupervisor: autoriza excepciones y revisa cierres.
  \item \TermAdmin: configura catálogos, impuestos, usuarios y políticas.
\end{itemize}

\chapter{Primeros Pasos}\label{ch:inicio}
\section{Acceso al sistema}
\SaulFigure{\ShotPath{inicio_sesion_01.png}}{Pantalla: Inicio de sesión}{fig:inicio-login}

\begin{enumerate}
  \item Ingrese usuario y contraseña.
  \item Verifique que la sucursal y caja sean correctas.
  \item Confirme que la fecha/hora del equipo sea correcta.
\end{enumerate}

\section{Checklist de configuración inicial}
\SaulFigure{\ShotPath{configuracion_inicial_01.png}}{Pantalla: Lista de configuración inicial}{fig:inicio-config}

\begin{longtable}{p{0.35\textwidth}p{0.59\textwidth}}
\toprule
\textbf{Elemento} & \textbf{Objetivo}\\
\midrule
\endhead
Moneda & Evitar diferencias de símbolos y decimales en tickets/reportes.\\
Impuestos & Asegurar cálculo legal y consistencia contable.\\
Pie de recibo & Incluir mensajes comerciales o fiscales requeridos.\\
Métodos de pago & Limitar opciones a las realmente aceptadas.\\
Usuarios y roles & Controlar qué puede hacer cada perfil.\\
\bottomrule
\end{longtable}

\section{Perfil de negocio y ajustes opcionales}
Si está habilitado por el administrador, complete datos fiscales del negocio, horarios y plantilla de recibo.

\chapter{Gestión de Catálogo}\label{ch:catalogo}
\section{Crear y editar productos}
\SaulFigure{\ShotPath{catalogo_producto_nuevo_01.png}}{Pantalla: Alta de producto}{fig:catalogo-producto}

Campos recomendados para cualquier producto:
\begin{itemize}
  \item Nombre claro.
  \item SKU interno.
  \item Código de barras.
  \item Categoría.
  \item Precio de venta y costo.
  \item Estado (activo/inactivo).
\end{itemize}

\section{Variantes, códigos de barras y categorías}
\SaulFigure{\ShotPath{catalogo_categorias_01.png}}{Pantalla: Gestión de categorías}{fig:catalogo-categorias}

Si está habilitado por el administrador, utilice variantes para talla/color. Cada variante debe tener identificador único.

\section{Listas de precios, promociones e importación}
Opciones usuales según configuración:
\begin{itemize}
  \item Lista de precio por canal o tipo de cliente.
  \item Promociones por periodo.
  \item Importación/exportación por archivo para altas masivas.
\end{itemize}

\chapter{Gestión de Clientes}\label{ch:clientes}
\section{Alta y búsqueda de clientes}
\SaulFigure{\ShotPath{clientes_lista_01.png}}{Pantalla: Listado y búsqueda de clientes}{fig:clientes-lista}

\begin{enumerate}
  \item Registre nombre y dato de contacto mínimo.
  \item Agregue identificación fiscal si aplica.
  \item Guarde y valide que el cliente aparezca en la búsqueda.
\end{enumerate}

\section{Historial y crédito (opcional)}
\SaulFigure{\ShotPath{clientes_historial_01.png}}{Pantalla: Historial de compras del cliente}{fig:clientes-historial}

Si el crédito está habilitado por el administrador, defina límite y plazo. Revise antigüedad de saldos semanalmente.

\chapter{Flujo de Ventas}\label{ch:ventas}
\section{Apertura de caja o turno (opcional)}
\SaulFigure{\ShotPath{ventas_abrir_caja_01.png}}{Pantalla: Apertura de caja}{fig:ventas-apertura}

\begin{enumerate}
  \item Registre fondo inicial.
  \item Verifique impresora y escáner.
  \item Confirme método para registrar gastos de caja.
\end{enumerate}

\section{Nueva venta: procedimiento completo}
\SaulFigure{\ShotPath{ventas_nueva_venta_01.png}}{Pantalla: Nueva venta}{fig:ventas-nueva}

\begin{enumerate}
  \item Seleccione la opción para crear una nueva venta.
  \item Agregue productos escaneando código o usando búsqueda.
  \item Ajuste cantidades según solicitud del cliente.
  \item Si corresponde, aplique descuento por línea o por ticket.
  \item Verifique impuestos y subtotal antes de cobrar.
  \item Agregue nota de operación cuando sea necesaria.
\end{enumerate}

\SaulFigure{\ShotPath{ventas_descuento_ticket_01.png}}{Pantalla: Descuento aplicado al ticket}{fig:ventas-descuento-ticket}

\subsection{Ejemplo numérico completo}
\begin{longtable}{p{0.40\textwidth}p{0.54\textwidth}}
\toprule
\textbf{Paso} & \textbf{Cálculo}\\
\midrule
\endhead
Subtotal inicial & 2 artículos de \$30.00 + 1 artículo de \$40.00 = \$100.00 \\
Descuento por ticket & 10\% de \$100.00 = \$10.00 \\
Subtotal con descuento & \$90.00 \\
Impuesto excluido 16\% & \$14.40 \\
Total a pagar & \$104.40 \\
\bottomrule
\end{longtable}

\section{Cobro y pagos divididos}
\SaulFigure{\ShotPath{ventas_checkout_pago_dividido_01.png}}{Pantalla: Cobro con pago dividido}{fig:ventas-pago-dividido}

\begin{enumerate}
  \item Ingrese monto por primer método (ej. efectivo).
  \item Ingrese monto restante por segundo método (ej. tarjeta).
  \item Verifique cambio a entregar en efectivo.
  \item Confirme y emita comprobante.
\end{enumerate}

\section{Recibo, ticket suspendido, cancelación y cierre}
\SaulFigure{\ShotPath{ventas_ticket_suspendido_01.png}}{Pantalla: Ticket suspendido}{fig:ventas-suspendido}
\SaulFigure{\ShotPath{ventas_cierre_turno_01.png}}{Pantalla: Cierre de turno}{fig:ventas-cierre}

\begin{warningbox}
La cancelación de venta debe conservar motivo y usuario responsable. Si está habilitado por el administrador, puede requerir autorización de supervisor.
\end{warningbox}

\subsection{Problemas frecuentes en venta}
\begin{longtable}{p{0.34\textwidth}p{0.60\textwidth}}
\toprule
\textbf{Situación} & \textbf{Acción recomendada}\\
\midrule
\endhead
El total no coincide con expectativa & Revise orden de descuentos y modo de impuesto (incluido/excluido).\\
No imprime ticket & Verifique conexión de impresora y cola de impresión; consulte Capítulo~\ref{ch:faq}.\\
Escáner no lee código & Limpie lector, confirme formato y pruebe búsqueda manual por SKU.\\
\bottomrule
\end{longtable}

\chapter{Devoluciones, Reembolsos y Cambios}\label{ch:devoluciones}
\section{Devolución con comprobante}
\SaulFigure{\ShotPath{devoluciones_por_recibo_01.png}}{Pantalla: Devolución por recibo}{fig:devoluciones-recibo}

\begin{enumerate}
  \item Busque la venta original por folio o fecha.
  \item Seleccione artículos a devolver (total o parcial).
  \item Defina estado del artículo devuelto (reintegrable/no reintegrable).
  \item Confirme método de reembolso y monto.
\end{enumerate}

\section{Devolución sin comprobante, parcial y cambios}
\SaulFigure{\ShotPath{devoluciones_cambio_01.png}}{Pantalla: Flujo de cambio de producto}{fig:devoluciones-cambio}

Si la política de devolución sin comprobante está habilitada por el administrador, aplique límites de monto y autorización.

\begin{warningbox}
No omita el rastro de auditoría: cada devolución debe registrar usuario, motivo, fecha y referencia de ticket cuando exista.
\end{warningbox}

\chapter{Operaciones de Inventario}\label{ch:inventario}
\section{Ajustes de stock}
\SaulFigure{\ShotPath{inventario_ajuste_stock_01.png}}{Pantalla: Ajuste de inventario}{fig:inventario-ajuste}

Use ajustes solo para corregir diferencias verificadas, no como reemplazo de recepciones o ventas.

\section{Recepción de compras}
\SaulFigure{\ShotPath{inventario_recepcion_compra_01.png}}{Pantalla: Recepción de compra}{fig:inventario-recepcion}

\begin{enumerate}
  \item Capture documento de compra.
  \item Compare unidades recibidas vs ordenadas.
  \item Registre faltantes/sobrantes.
  \item Confirme ingreso a inventario.
\end{enumerate}

\section{Conteos físicos y alertas}
\SaulFigure{\ShotPath{inventario_conteo_fisico_01.png}}{Pantalla: Conteo físico de inventario}{fig:inventario-conteo}

Si las alertas de mínimo están habilitadas por el administrador, revise diariamente los productos críticos.

\chapter{Reportes y Analítica}\label{ch:reportes}
\section{Reportes operativos principales}
\SaulFigure{\ShotPath{reportes_resumen_ventas_01.png}}{Pantalla: Resumen de ventas}{fig:reportes-resumen}
\SaulFigure{\ShotPath{reportes_impuestos_01.png}}{Pantalla: Reporte de impuestos}{fig:reportes-impuestos}

Reportes recomendados por cierre diario:
\begin{itemize}
  \item Resumen de ventas por periodo.
  \item Productos más vendidos.
  \item Reporte de descuentos.
  \item Reporte de impuestos.
  \item Desempeño por cajero.
\end{itemize}

\section{Exportación de reportes}
Si la exportación está habilitada por el administrador, utilice formatos estándar (por ejemplo CSV o PDF) para auditoría y respaldo.

\section{Indicadores recomendados para gestión}
Use un tablero semanal con indicadores simples y repetibles:
\begin{longtable}{p{0.32\textwidth}p{0.30\textwidth}p{0.32\textwidth}}
\toprule
\textbf{Indicador} & \textbf{Fórmula base} & \textbf{Decisión que soporta}\\
\midrule
\endhead
Ticket promedio & Ventas netas / Número de tickets & Detectar oportunidades de venta sugerida.\\
Margen bruto estimado & (Ventas netas - Costo estimado) / Ventas netas & Medir rentabilidad operativa.\\
Descuento sobre venta & Total descuentos / Ventas brutas & Controlar disciplina comercial.\\
Tasa de devolución & Monto devuelto / Ventas netas & Detectar problemas de calidad o despacho.\\
\bottomrule
\end{longtable}

\begin{notebox}
No compare indicadores entre periodos con políticas distintas de descuento o impuesto. Para análisis histórico, mantenga reglas estables.
\end{notebox}

\chapter{Administración y Configuración}\label{ch:admin}
\section{Usuarios y roles}
\SaulFigure{\ShotPath{admin_usuarios_roles_01.png}}{Pantalla: Usuarios y roles}{fig:admin-roles}

Buenas prácticas:
\begin{itemize}
  \item Cada usuario con credenciales individuales.
  \item Evitar cuentas compartidas.
  \item Revisar permisos al menos una vez por mes.
\end{itemize}

\section{Tienda, impuestos, recibos e integraciones}
\SaulFigure{\ShotPath{admin_configuracion_tienda_01.png}}{Pantalla: Configuración general de tienda}{fig:admin-tienda}

Configure moneda, zona horaria, impuestos y plantillas de recibo. Las integraciones externas son opcionales y dependen de licenciamiento y configuración.

\chapter{Solución de Problemas y FAQ}\label{ch:faq}
\section{Incidencias frecuentes}
\SaulFigure{\ShotPath{faq_diagnostico_impresora_01.png}}{Pantalla: Diagnóstico de impresora}{fig:faq-impresora}

\begin{longtable}{p{0.34\textwidth}p{0.60\textwidth}}
\toprule
\textbf{Problema} & \textbf{Respuesta rápida}\\
\midrule
\endhead
No puedo iniciar sesión & Verifique usuario/clave, mayúsculas y estado de red local.\\
La impresora no responde & Revise energía, cableado y dispositivo predeterminado.\\
No se lee el código de barras & Verifique etiqueta, simbología y limpieza del lector.\\
Totales no coinciden & Revise redondeo, orden de descuento e impuestos configurados.\\
No permite reembolso & Puede estar fuera de ventana de devolución o sin permiso de rol.\\
\bottomrule
\end{longtable}

\section{Respaldo y desempeño}
\begin{tipbox}
Programe respaldos diarios y valide restauración al menos una vez al mes.
\end{tipbox}

\begin{notebox}
Si el sistema se vuelve lento, revise cantidad de programas abiertos, estado de red local y periodo de reportes seleccionado.
\end{notebox}

\chapter{Operación por Roles y Control Interno}\label{ch:roles}
\section{Rutina operativa del cajero}
Propuesta de rutina por turno:
\begin{enumerate}
  \item Confirmar apertura correcta de caja y fondo inicial.
  \item Validar que impresora, escáner y métodos de pago estén disponibles.
  \item Registrar ventas evitando usuarios compartidos.
  \item Escalar excepciones (descuentos altos, devoluciones sin ticket) al supervisor.
  \item Ejecutar pre-cierre y conciliación antes del cierre final.
\end{enumerate}

\section{Matriz de aprobaciones del supervisor}
\SaulFigure{\ShotPath{roles_matriz_aprobaciones_01.png}}{Pantalla: Matriz de aprobaciones del supervisor}{fig:roles-matriz}

\begin{longtable}{p{0.34\textwidth}p{0.24\textwidth}p{0.36\textwidth}}
\toprule
\textbf{Evento} & \textbf{Nivel sugerido} & \textbf{Control mínimo}\\
\midrule
\endhead
Descuento superior al límite normal & Supervisor & Motivo obligatorio y usuario autorizador.\\
Cancelación posterior al cobro & Supervisor o administrador & Referencia al ticket y bitácora de cambios.\\
Devolución sin comprobante & Supervisor & Límite de monto + identificación del cliente.\\
Ajuste de inventario de alto impacto & Administrador & Doble validación y evidencia física.\\
\bottomrule
\end{longtable}

\section{Gobierno de catálogos por administración}
Recomendación para evitar errores masivos:
\begin{itemize}
  \item Definir horario de mantenimiento para cambios de precio.
  \item Congelar cambios durante horas pico.
  \item Validar en ambiente de prueba cuando exista.
  \item Documentar fecha, responsable y motivo de cada ajuste masivo.
\end{itemize}

\section{Revisión de bitácora y trazabilidad}
\SaulFigure{\ShotPath{roles_bitacora_auditoria_01.png}}{Pantalla: Bitácora de auditoría}{fig:roles-bitacora}

\begin{warningbox}
Una operación sin trazabilidad (sin usuario, motivo o referencia) debe tratarse como incidente de control interno.
\end{warningbox}

\chapter{Procedimientos Estándar (SOP)}\label{ch:sop}
\section{SOP 01: Venta con fila de clientes}
\begin{enumerate}
  \item Mantenga una sola venta activa por cajero.
  \item Escanee, confirme cantidad y anuncie subtotal antes de cobrar.
  \item Si el cliente solicita cambios, detenga el cobro y confirme nuevamente el total.
  \item Entregue ticket y agradezca antes de iniciar el siguiente cliente.
\end{enumerate}

\section{SOP 02: Diferencia de precio en caja}
\begin{enumerate}
  \item Detenga temporalmente la operación.
  \item Verifique precio del catálogo y evidencia en anaquel.
  \item Aplique política comercial vigente (precio exhibido o precio de sistema según norma interna).
  \item Registre nota con motivo para seguimiento de catálogo.
\end{enumerate}

\section{SOP 03: Contingencia de red o servicios}
\SaulFigure{\ShotPath{sop_contingencia_red_01.png}}{Pantalla: Procedimiento de contingencia por falla de red}{fig:sop-red}

\begin{enumerate}
  \item Identifique si la falla afecta solo una caja o toda la sucursal.
  \item Notifique al supervisor y active protocolo local.
  \item Si está habilitado por el administrador, continúe con modo contingencia.
  \item Al restablecer servicio, reconcilie operaciones pendientes.
\end{enumerate}

\begin{warningbox}
Nunca duplique manualmente cobros ya registrados. Antes de repetir una transacción, confirme estado final del ticket.
\end{warningbox}

\section{SOP 04: Cierre diario con múltiples cajas}
\begin{enumerate}
  \item Ejecute pre-cierre por caja.
  \item Consolide ventas, descuentos, devoluciones y retiros.
  \item Documente diferencias y responsables.
  \item Cierre cada caja y luego cierre de sucursal.
  \item Exportar reportes clave (ventas, impuestos, descuentos, devoluciones).
\end{enumerate}

\begin{tipbox}
Una reunión breve de 10 minutos al final del día acelera correcciones para el siguiente turno.
\end{tipbox}

\chapter{Glosario}\label{ch:glosario}
\begin{SaulGlossary}
\GlossaryEntry{Ajuste de inventario}{Corrección manual de existencias por diferencia detectada.}
\GlossaryEntry{Barcode / Código de barras}{Representación gráfica para identificar productos al escanear.}
\GlossaryEntry{Categoría}{Agrupación lógica de productos para búsqueda y reportes.}
\GlossaryEntry{Costo}{Valor interno de adquisición o producción del artículo.}
\GlossaryEntry{Descuento}{Reducción del precio aplicada por línea o por ticket.}
\GlossaryEntry{Impuesto incluido}{Modalidad donde el precio ya contiene impuesto.}
\GlossaryEntry{Impuesto excluido}{Modalidad donde el impuesto se suma al final.}
\GlossaryEntry{Margen}{Relación entre precio de venta y costo.}
\GlossaryEntry{Pago dividido}{Cobro de una venta con más de un método de pago.}
\GlossaryEntry{Recepción}{Ingreso de mercancía al inventario desde una compra.}
\GlossaryEntry{Recibo / Ticket}{Comprobante de venta entregado al cliente.}
\GlossaryEntry{SKU}{Identificador interno único de un producto.}
\GlossaryEntry{Stock disponible}{Cantidad actual vendible de un producto.}
\GlossaryEntry{Ticket suspendido}{Venta guardada temporalmente para continuar después.}
\GlossaryEntry{Trazabilidad}{Capacidad de auditar quién hizo qué, cuándo y por qué.}
\end{SaulGlossary}

\appendix
\chapter{Checklist de Capturas de Pantalla}\label{app:capturas}
Todas las capturas deben guardarse dentro de \texttt{manual/images/placeholders/es/} con los nombres indicados.

\begin{longtable}{p{0.38\textwidth}p{0.44\textwidth}p{0.12\textwidth}}
\toprule
\textbf{Archivo} & \textbf{Qué debe mostrar} & \textbf{Prioridad}\\
\midrule
\endhead
\ShotRef{introduccion_panel_principal_01.png} & Panel principal del sistema. & Alta\\
\ShotRef{conceptos_descuento_item_01.png} & Aplicación de descuento por línea. & Alta\\
\ShotRef{inicio_sesion_01.png} & Formulario de acceso. & Alta\\
\ShotRef{configuracion_inicial_01.png} & Lista de ajustes iniciales. & Media\\
\ShotRef{catalogo_producto_nuevo_01.png} & Formulario alta/edición de producto. & Alta\\
\ShotRef{catalogo_categorias_01.png} & Gestión de categorías y subcategorías. & Media\\
\ShotRef{clientes_lista_01.png} & Búsqueda y listado de clientes. & Alta\\
\ShotRef{clientes_historial_01.png} & Historial de compras del cliente. & Media\\
\ShotRef{ventas_abrir_caja_01.png} & Apertura de caja o turno. & Media\\
\ShotRef{ventas_nueva_venta_01.png} & Flujo principal de nueva venta. & Alta\\
\ShotRef{ventas_descuento_ticket_01.png} & Aplicación de descuento al ticket. & Alta\\
\ShotRef{ventas_checkout_pago_dividido_01.png} & Pantalla de cobro con split payment. & Alta\\
\ShotRef{ventas_ticket_suspendido_01.png} & Ticket suspendido/reanudación. & Media\\
\ShotRef{ventas_cierre_turno_01.png} & Cierre de caja. & Media\\
\ShotRef{devoluciones_por_recibo_01.png} & Devolución con comprobante. & Alta\\
\ShotRef{devoluciones_cambio_01.png} & Flujo de cambio de producto. & Media\\
\ShotRef{inventario_ajuste_stock_01.png} & Ajuste de inventario. & Alta\\
\ShotRef{inventario_recepcion_compra_01.png} & Recepción de compra. & Alta\\
\ShotRef{inventario_conteo_fisico_01.png} & Conteo físico. & Media\\
\ShotRef{reportes_resumen_ventas_01.png} & Dashboard/resumen de ventas. & Alta\\
\ShotRef{reportes_impuestos_01.png} & Reporte de impuestos. & Alta\\
\ShotRef{admin_usuarios_roles_01.png} & Matriz de usuarios y permisos. & Alta\\
\ShotRef{admin_configuracion_tienda_01.png} & Parámetros generales de tienda. & Alta\\
\ShotRef{faq_diagnostico_impresora_01.png} & Diagnóstico de impresora. & Media\\
\ShotRef{roles_matriz_aprobaciones_01.png} & Matriz de aprobaciones y excepciones por rol. & Alta\\
\ShotRef{roles_bitacora_auditoria_01.png} & Bitácora de auditoría de operaciones críticas. & Alta\\
\ShotRef{sop_contingencia_red_01.png} & Protocolo de continuidad por caída de red. & Media\\
\ShotRef{soporte_contacto_01.png} & Vista de ayuda o contacto de soporte. & Baja\\
\bottomrule
\end{longtable}

\chapter{Referencia Rápida}\label{app:rapida}
\textbf{Abrir una venta en menos de 30 segundos}
\begin{enumerate}
  \item Iniciar sesión y confirmar caja activa.
  \item Crear nueva venta.
  \item Escanear artículos.
  \item Confirmar total y cobrar.
  \item Emitir ticket.
\end{enumerate}

\textbf{Devolver un producto con recibo}
\begin{enumerate}
  \item Buscar ticket original.
  \item Seleccionar artículo y motivo.
  \item Confirmar método de reembolso.
  \item Guardar operación con trazabilidad.
\end{enumerate}

\textbf{Cierre diario}
\begin{enumerate}
  \item Revisar ventas del día.
  \item Conciliar efectivo vs sistema.
  \item Ejecutar cierre de turno.
  \item Exportar reportes clave (ventas, descuentos, impuestos).
\end{enumerate}

\chapter{Atajos de teclado (opcional)}\label{app:atajos}
Estos atajos pueden variar según versión o configuración.

\begin{longtable}{p{0.24\textwidth}p{0.70\textwidth}}
\toprule
\textbf{Atajo} & \textbf{Acción esperada}\\
\midrule
\endhead
\texttt{F2} & Iniciar nueva venta.\\
\texttt{F4} & Buscar producto.\\
\texttt{F6} & Aplicar descuento (si está habilitado y autorizado).\\
\texttt{F8} & Ir a cobro.\\
\texttt{Ctrl+P} & Reimprimir último ticket (si está permitido).\\
\bottomrule
\end{longtable}

\chapter{Soporte y Contacto}\label{app:soporte}
\SaulFigure{\ShotPath{soporte_contacto_01.png}}{Pantalla: Soporte y contacto}{fig:soporte-contacto}

Campos sugeridos para completar en su organización:
\begin{itemize}
  \item Correo de soporte: \texttt{[soporte@tu-dominio.com]}
  \item Teléfono de soporte: \texttt{[+00 0000 0000]}
  \item Horario de atención: \texttt{[L-V 09:00-18:00]}
  \item Tiempo objetivo de respuesta: \texttt{[ej. 4 horas hábiles]}
\end{itemize}

\begin{notebox}
Para incidentes críticos, registre: fecha, sucursal, caja, usuario, pasos para reproducir y evidencia (captura o folio).
\end{notebox}

\end{document}
